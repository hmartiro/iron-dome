\documentclass[12pt,a4paper,notitlepage]{report}
\usepackage[utf8]{inputenc}
\usepackage{amsmath}
\usepackage{amsfonts}
\usepackage{amssymb}
\begin{document}
\section*{Projectile Interception}
The robot intercepts projectiles on a floating window in front of it, which is
a bounded section of a sphere. For each projectile the robot can see, the trajectory is 
predicted based on the observations from the Kinect depth image, which are fed into a Kalman filter
to deal with the noise and any problems with detection of the ball. The intersection point is found by solving the following system of equations:
\begin{align*}
x &= x_0+v_x*t+\frac{1}{2}a_x^2\\
y &= y_0+v_y*t+\frac{1}{2}a_y^2\\
z &= z_0+v_z*t+\frac{1}{2}a_z^2\\
R^2 &= (x_s-x)^2+(y_s-y)^2+(z_s-z)^2
\end{align*}

Where \(x,y,z\) are the position of the ball with respect to the robot base frame.
The acceleration in z is assumed to be \(-9.81 \frac{m}{s}\) while in x and y it
 is assumed to be zero. The radius \(R\) and position of the sphere are chosen such that the window
 in front of the robot is within the usable workspace of the robot. We used a polynomial solver 
 on the fourth order equation in \(t\) to find the time and location
  that the predicted trajectory would intercept the sphere. If one or more real roots exist, then
  there will be an intersection at the time of the lowest real root.
  
\end{document}